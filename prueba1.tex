% Created 2026-01-06 Tue 21:22
% Intended LaTeX compiler: pdflatex
\documentclass[10pt, aspectratio=169]{beamer}
\usepackage[utf8]{inputenc}
\usepackage[T1]{fontenc}
\usepackage{graphicx}
\usepackage{longtable}
\usepackage{wrapfig}
\usepackage{rotating}
\usepackage[normalem]{ulem}
\usepackage{amsmath}
\usepackage{amssymb}
\usepackage{capt-of}
\usepackage{hyperref}
\usetheme{Madrid}
\definecolor{TuAzul}{RGB}{0,85,127}
\usecolortheme[named=TuAzul]{structure}
\setbeamercolor{frametitle}{bg=TuAzul, fg=white}
\setbeamercolor{title}{bg=TuAzul, fg=white}
\setbeamertemplate{navigation symbols}{}
\usetheme{default}
\author{Ricardo León}
\date{2026-01-06}
\title{PROYECTO FINAL}
\hypersetup{
 pdfauthor={Ricardo León},
 pdftitle={PROYECTO FINAL},
 pdfkeywords={},
 pdfsubject={},
 pdfcreator={Emacs 29.3 (Org mode 9.6.15)}, 
 pdflang={English}}
\begin{document}

\maketitle

\section{Contexto Histórico}
\label{sec:orgc2410c7}
\begin{frame}[label={sec:org7f6793b}]{La Batalla de Pichincha}
La Batalla de Pichincha (\alert{\alert{24 de mayo de 1822}}) selló la independencia de Ecuador.

\begin{itemize}
\item \alert{Lugar:} Faldas del volcán Pichincha, Quito.
\item \alert{Comandante Patriota:} Antonio José de Sucre.
\item \alert{Comandante Realista:} Melchor Aymerich.
\end{itemize}
\end{frame}

\begin{frame}[label={sec:org3429eba}]{Antecedentes del Conflicto}
Antes de llegar a Quito, las tropas libertadoras tuvieron que:
\begin{enumerate}
\item Avanzar desde Guayaquil hacia la sierra.
\item Cruzar la cordillera de los Andes con difícil clima.
\item Vencer en batallas previas como la de Riobamba.
\end{enumerate}
\end{frame}

\section{Análisis Estratégico}
\label{sec:org3a2450f}
\begin{frame}[label={sec:org116d101}]{Estrategia Militar (Columnas)}
\begin{columns}
\begin{column}{0.6\columnwidth}
Sucre ordenó un ascenso nocturno para sorprender al enemigo por la retaguardia.

\begin{itemize}
\item La maniobra fue arriesgada por el terreno.
\item El batallón Albión jugó un papel crucial al flanquear a los realistas.
\end{itemize}
\end{column}

\begin{column}{0.4\columnwidth}
\centering
\Huge [MAPA]
\normalsize
\vspace{0.5cm}
\emph{Esquema táctico}
\end{column}
\end{columns}
\end{frame}

\begin{frame}[label={sec:org59952c0}]{Resultados y Consecuencias}
\begin{block}{Impacto Inmediato}
La capitulación de Aymerich y la entrada triunfal en Quito.
\end{block}

\begin{alertblock}{Legado}
Quito se unió a la Gran Colombia y sirvió de base para liberar Perú.
\end{alertblock}
\end{frame}

\section{Cierre}
\label{sec:org43f5da5}
\begin{frame}[label={sec:org246f83e}]{}
\vspace{1.5cm}
\centering
\Huge \textbf{¡Muchas Gracias!}

\vspace{1cm}
\large ¿Preguntas?

\vspace{1cm}
\footnotesize ricardo.leon02@epn.edu.ec
\end{frame}
\end{document}
